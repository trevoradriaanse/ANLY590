\documentclass[11pt]{article}
\linespread{1.5}

%%%%%%%%%%%%%
%%% PACKAGES %%%
%%%%%%%%%%%%%

\usepackage[utf8]{inputenc}
\usepackage[english]{babel}

\usepackage{listings}
\usepackage[most]{tcolorbox}
\usepackage{inconsolata}
\newtcblisting[auto counter]{sexylisting}[2][]{sharp corners, 
    fonttitle=\bfseries\footnotesize, colframe=gray, listing only, 
    listing options={basicstyle=\ttfamily\footnotesize,language=matlab}, 
    title= #2, #1}

\usepackage{amsmath,amsthm, amsfonts, latexsym, amssymb, mathtools,tikz, tikz-qtree}
\tikzset{every tree node/.style={align=center, anchor=north}}
\usepackage{graphicx,color, mathrsfs, enumitem, fancyhdr, multicol, upgreek, fancybox}
\usepackage[b]{esvect}
\usepackage{libertine}
\usepackage[libertine]{newtxmath}
\usepackage{gb4e, cgloss4e} 
 \usepackage{tipa} %textipa{}
 \usepackage{wrapfig}

\usepackage{alltt,xcolor}

%%%%%%%%%%%
%%% HEADER %%%
%%%%%%%%%%%

\newcommand{\name}{Trevor Adriaanse, Jack Hart, Norman Hong}
\newcommand{\Date}{Due 12/6/16}
\newcommand{\topic}{ANLY 590}
\newcommand{\course}{MATH 343}

\newcommand{\myheader}[1]{
\newpage
\noindent
%{\Large {\bf \hfill Some notes on Linguistics\hfill}}
{\bf \name \hfill \topic \vspace{1mm}}
\hrule \medskip
}

%%%%%%%%%%%%%%%%%
%%% FOOTER/MARGINS %%%
%%%%%%%%%%%%%%%%%

\fancyfoot[C]{\changefont \thepage}
\setlength{\parskip}{.5pc}
\setlength{\parindent}{0pt}
\setlength{\topmargin}{-5.5pc}
\setlength{\textheight}{9.25in}
\setlength{\oddsidemargin}{0pc}
\setlength{\evensidemargin}{0pc}
\setlength{\textwidth}{6.5in}
\newcommand{\changefont}{
    \fontsize{7}{7}\selectfont
}
\fancyhf{}

%%%%%%%%%%%%%%%%
%%% NEW DOCUMENT %%%
%%%%%%%%%%%%%%%%

\begin{document} 

\myheader

{\large {\bf Initial Project Proposal: Computer Vision for Cancer Detection}}

~~~~We propose using deep learning to perform binary classification on the PatchCamelyon (PCam) dataset^{\text{[1][2]}}. Specifically, the related Kaggle problem^{\text{[3]}} is concerned with identifying cancer in image scans of lymph nodes.

~~~~Histopathologic images are images used to study disease in tissues. The PCam dataset contains histopathologic scans from lymph node sections in the breasts with binary labels specifying presence of cancer. These histology slides are obtained from surgical specimens placed onto glass slides after processing and staining. Lymph nodes are often the first place breast cancer metastasizes to, and histological assessment is a diagnostic procedure used to determine whether the tissue has been afflicted by cancer. Automating approaches to detect lymph node metastasis is an attractive avenue of exploration to aid the pathologist in their workload, where misinterpretation and tediousness can lead to incorrect diagnoses.

~~~~Convolutional neural networks are commonly applied to computer vision, to confront tasks of the human visual system. We will explore modern neural approaches to image analysis and construct a network capable of performing well at the classification objective. 

~~~~This project brings together the advances made in digital pathology yielding the high quality color images of the PCam dataset with strides made in the deep learning community to perform accurate and efficient computer vision. In building a deep learning model for this classification problem, we hope to learn more about neural approaches to perceptual tasks like image analysis and gain an appreciation for how solving these problems can have positive outcomes in the world.

\begin{thebibliography}{9}

\bibitem{2}
B. S. Veeling, J. Linmans, J. Winkens, T. Cohen, M. Welling.
\textit{Rotation Equivariant CNNs for Digital Pathology}.
arXiv:1806.03962.

\bibitem{2}
Ehteshami Bejnordi et al.
\textit{Diagnostic Assessment of Deep Learning Algorithms for Detection of Lymph Node Metastases in Women With Breast Cancer}.
JAMA: The Journal of the American Medical Association, 318(22), 2199–2210. doi:jama.2017.14585.

\bibitem{1}
https://www.kaggle.com/c/histopathologic-cancer-detection


\end{thebibliography}


\end{document}








